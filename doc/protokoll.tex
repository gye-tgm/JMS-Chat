\documentclass[11pt, a4paper]{article}

\usepackage[utf8]{inputenc}
\usepackage{listings}             % Include the listings-package for c++,java,idl
% \usepackage[ngerman]{babel}

\title{JMS-Chat}
\author{Elias Frantar, Gary Ye (4BHIT)}
\date{\today{}, Wien}
\begin{document}

% Pretty Code formatting
\lstset{basicstyle=\ttfamily\small,
        keywordstyle=,
        commentstyle=\itshape,
        numbers=left,                   % where to put the line-numbers
        stepnumber=1,					% line number incrementing
        linebreak=true,					% line wrapping on
        numberstyle=\tiny,				% line number style	
        showstringspaces=false,			
        abovecaptionskip=0pt,
        belowcaptionskip=0pt,
        xleftmargin=\parindent,
        fontadjust}

\maketitle
\newpage
\tableofcontents
\newpage

\section{Aufgabenstellung}

\section{Aufwandsschätzung}
Installation Message Broker Apache ActiveMQ wurde im Unterricht durchgeführt und entfällt daher.

Bekanntmachung mit JMS 1.5h
Planung 1.5h

Frantar
Implementierung JMSChat 1.5h
Implementierung JMSMail 1.5h

Ye
Impl SyntaxChecker 0.75h
SyntaxCheckerTest 0.5h
Impl User 2h

Gesamttesting 1.5

Task Elias Frantar Gary Ye



Implementierung des Chatraums (JMS Topic) 3h
Implementierung der Postfach-Funktionalität (JMS Queue) 3h
Dokumentation 3h
Testing 2h
\section{Installation}

\begin{enumerate}
\item Herunterladen der Installationsdateien[1].
\item Entpacken der Installationsdateien mit unzip
\item Ändern der Rechte von bin/activemq mit chmod u+x, damit die Datei ausführbar wird.
\end{itemize}


\section{Aufgetretene Probleme}
\subsection{Installation}
Für das Debian-Image muss eine Änderung durchgeführt werden, damit Apache Active MQ gestartet werden kann.
> Öffnen Sie die Datei bin/activemq mit einem Editor Ihrer Wahl 
> Suchen Sie nach folgender Zeile
       ACTIVEMQ_OPTS_MEMORY="-Xms1G -Xmx1G"
> Ersetzen Sie diese Zeile mit:
       ACTIVEMQ_OPTS_MEMORY="-Xms512M -Xmx512M"
\section{Testing}



\end{document}
