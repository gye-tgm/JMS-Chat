\documentclass[11pt, a4paper]{article}

\usepackage[utf8]{inputenc}
\usepackage{listings}             % Include the listings-package for c++,java,idl
% \usepackage[ngerman]{babel}

\title{JMS-Chat}
\author{Elias Frantar, Gary Ye (4BHIT)}
\date{\today{}, Wien}
\begin{document}

% Pretty Code formatting
\lstset{basicstyle=\ttfamily\small,
        keywordstyle=,
        commentstyle=\itshape,
        numbers=left,                   % where to put the line-numbers
        stepnumber=1,					% line number incrementing
        breaklines=true,					% line wrapping on
        numberstyle=\tiny,				% line number style	
        showstringspaces=false,			
        abovecaptionskip=0pt,
        belowcaptionskip=0pt,
        xleftmargin=\parindent,
        fontadjust}

\maketitle
\newpage
\tableofcontents
\newpage

\section{Aufgabenstellung}

\section{Aufwandsschätzung}
Installation Message Broker Apache ActiveMQ wurde im Unterricht durchgeführt und entfällt daher.

\begin{description}
\item[Alle Teammitglieder] \hfill \\

  \begin{tabular} {| l | l |}
    \hline
    Task & Tatsächlich \\ \hline
    Bekanntmachung mit JMS & 1.5h \\ \hline
    Planung & 1.5h \\ \hline
    \hline
    Gesamt & 3h \\ \hline
  \end{tabular}

\item[Frantar] \hfill \\

  \begin{tabular} {| l | l | l |}
    \hline
    Task & Schätzung & Tatsächlich \\ \hline
    Implementierung JMSChat & 1.5h & 0.75h \\ \hline
    Implementierung JMSMail & 1.5h & 1.5h \\ \hline
    \hline
    Gesamt & 3h & 2.75h \\ \hline
  \end{tabular}

\item[Ye] \hfill \\

    \begin{tabular} {| l | l | l |}
      \hline
      Task & Schätzung & Aufwand \\ \hline
      SyntaxChecker Implementierung & 0.75h & 0.25h \\ \hline
      SyntaxChecker Test & 0.5h & 0.5h \\ \hline
      JMS-User Implementierung & 2h & 0.5h \\ \hline
      \hline
      Gesamt & 3.25h & 1.25h \\ \hline
    \end{tabular}

\end{description}

\section{Installation}
\begin{enumerate}
\item Herunterladen der Installationsdateien[1].
\item Entpacken der Installationsdateien mit unzip
\item Ändern der Rechte von bin/activemq mit chmod u+x, damit die Datei ausführbar wird.
\end{enumerate}

\section{Aufgetretene Probleme}
\subsection{Installation}
Für das Debian-Image muss eine Änderung durchgeführt werden, damit Apache Active MQ gestartet werden kann.
Öffnen Sie die Datei bin/activemq mit einem Editor Ihrer Wahl 
Suchen Sie nach folgender Zeile
%       ACTIVEMQ_OPTS_MEMORY="-Xms1G -Xmx1G"
Ersetzen Sie diese Zeile mit:
%       ACTIVEMQ_OPTS_MEMORY="-Xms512M -Xmx512M"
\section{Testing}



\end{document}
